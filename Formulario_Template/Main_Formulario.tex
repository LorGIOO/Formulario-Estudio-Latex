%Archivo Main

\documentclass[a4paper,10pt]{report}

% Paquetes esenciales
\usepackage{graphicx}    % Inclusión de gráficos
\usepackage{geometry}    % Configuración de márgenes
\usepackage{fancyhdr}    % Encabezados y pies de página personalizados
\usepackage{titlesec}    % Personalización de títulos
\usepackage{fontspec}    % Selección de fuentes (compatible con XeLaTeX y LuaLaTeX)
\usepackage{xcolor}      % Colores
\usepackage{tabularx}    % Paquete para tablas flexibles
\usepackage{tocloft}     % Personalización del índice
\usepackage{amsmath}
\usepackage{array}
\usepackage{booktabs}
\usepackage{amsfonts}
\usepackage{amssymb}
\usepackage{ragged2e}    % Para justificar el texto
\usepackage{setspace}    % Para espaciar entre palabras y líneas
\usepackage[colorlinks=true, linkcolor=black, pdfborder={0 0 0}]{hyperref} % Hipervínculos
\usepackage{longtable}   % Tablas largas

% Configuración de márgenes
\geometry{left=1.3cm, right=1.3cm, top=2cm, bottom=1.3cm}

% Configuración de la fuente del documento
\setmainfont{Montserrat}

% Definir color personalizado
\definecolor{custompurple}{HTML}{5C068C}

% Configuración de sangría global
\setlength{\parindent}{0pt}

% Ajuste de espaciado de línea
\setstretch{1.25}

% Establecer la profundidad del índice
\setcounter{tocdepth}{3}

\renewcommand{\contentsname}{Índice}
\makeatletter
\renewcommand{\tableofcontents}{
	\begin{center}
		{\Large\bfseries \contentsname} % Utiliza el nombre redefinido "Índice"
	\end{center}
	\vspace{-0.5cm} % Ajusta el espacio entre el título y el contenido del índice
	\@starttoc{toc}
}
\makeatother

% Estilo para títulos
\titleformat{\section}[block]{\Large\bfseries\centering}{}{0pt}{}
\titleformat{\subsection}[block]{\normalsize\normalfont}{}{0pt}{}  % Evitar negrita en subsecciones

% Configuración del índice
\renewcommand{\cftsecleader}{\cftdotfill{\cftdotsep}}

% Definición de \sectionnc
\newcommand{\sectionnc}[1]{%
	\phantomsection
	\section*{#1}
	\addcontentsline{toc}{section}{#1}
}

% Definición de \subsectionnc
\newcommand{\subsectionnc}[1]{%
	\phantomsection
	\addcontentsline{toc}{subsection}{#1}%
	\makebox[0pt][l]{#1}% Mantiene el formato normal y alineado
}

% Definición del nuevo comando \formula sin negrita
\newcommand{\formula}[3]{%
	\phantomsection
	\addcontentsline{toc}{subsection}{#1}%
	\noindent
	\begin{minipage}[t]{0.3\textwidth} 
		\raggedright #1
	\end{minipage}% 
	\begin{minipage}[t]{0.4\textwidth} 
		\centering #2
	\end{minipage}% 
	\begin{minipage}[t]{0.3\textwidth}
		\raggedleft #3
	\end{minipage}
	
	\bigskip
}


% Estilo de portada (logo)
\fancypagestyle{titlepage}{
	\fancyhf{}
	\fancyhead[L]{\hspace{0cm}\raisebox{-0.3cm}{\includegraphics[width=9cm]{./999. recursos/Secciones/seccion-ingenieria-civil-original.png}}} % Ajusta la posición del logo de la portada
	\renewcommand{\headrulewidth}{0pt}
}

% Asegúrate de tener el paquete tabularx cargado
\usepackage{tabularx}

% Estilo del índice (utilizando el mismo estilo que las páginas principales)
\fancypagestyle{indexpage}{
	\fancyhf{}
	\fancyhead[L]{\hspace{0cm}\raisebox{3cm}{\includegraphics[width=1.5cm]{./999. recursos/icono-ull-original.png}}} % Ajuste del logo en el índice
	\fancyfoot[R]{\thepage} % Número de página alineado al margen derecho
	\vspace{0.2cm} % Separación mínima entre logo y texto
	\renewcommand{\headrulewidth}{0pt}
	\renewcommand{\footrulewidth}{0pt}
}

% Estilo para el resto del documento (a partir de 'Tema n')
\fancypagestyle{mainmatter}{
	\fancyhf{}
	\fancyhead[L]{\includegraphics[width=1.5cm]{./999. recursos/icono-ull-original.png}}
	\fancyfoot[R]{\thepage}
	\renewcommand{\headrulewidth}{0pt}
	\renewcommand{\footrulewidth}{0pt}
}

% Establecer el estilo de página por defecto
\pagestyle{mainmatter}

% Documento principal
\begin{document}
	
	% Portada
	\begin{titlepage}
		\thispagestyle{titlepage}
		\begin{titlepage}
	\thispagestyle{titlepage} % Aplica el estilo con el logo de Sección de Ingeniería Civil
	\vspace*{8cm}
	\begin{center}
		% Título principal con tamaño 28pt y color personalizado, centrado
		{\fontsize{28pt}{28pt}\selectfont\textbf{\textcolor{custompurple}{Formulario}}}
		
		\vspace{0.5cm} % Espacio entre el título y subtítulo
		
		% Subtítulo con tamaño 24pt y el mismo color
		{\fontsize{24pt}{24pt}\selectfont\textcolor{custompurple}{Asignatura}}
		
		\vspace{1cm}
		
		% Nombre y correo electrónico sin resaltado
		\textbf{Nombre Apellido 1 Apellido 2} - alu0000000000@ull.edu.es
	\end{center}
	\vfill
\end{titlepage}
 % Ruta relativa para la portada
	\end{titlepage}
	
	% Página en blanco sin numeración
	\newpage
	\thispagestyle{empty}
	\mbox{}
	
	\newpage
	% Índice con estilo personalizado
	\pagenumbering{gobble} % Sin numeración visible
	\begin{titlepage}
		\thispagestyle{indexpage}
		% Indice.tex


\tableofcontents % Genera el índice
\thispagestyle{indexpage} % Aplica el estilo con el logo en la esquina superior izquierda
\newpage % Añade una nueva página después del índice
 % Ruta relativa para el índice
	\end{titlepage}
	
	% Nomenclatura
	\newpage
	\pagenumbering{arabic} % Comienza numeración arábiga
	\sectionnc{Nomenclatura}

\begin{center}
	\renewcommand{\arraystretch}{1.1}
	\begin{tabular}{
			>{\centering\arraybackslash}m{2cm} | 
			>{\raggedright\arraybackslash}m{8cm} | 
			>{\centering\arraybackslash}m{3cm}   
		}
		\textbf{Símbolo} & \textbf{Nombre} & \textbf{Unidades} \\ \hline
		\hypertarget{Q}{$Q$} & Caudal & $\frac{\text{m}^3}{\text{s}}$ \\ 
		\hypertarget{u}{$u$} & Velocidad & $\frac{\text{m}}{\text{s}}$ \\ 
		\hypertarget{A}{$A$} & Área & $\text{m}^2$ \\ 
		\hypertarget{V}{$V$} & Volumen & $\text{m}^3$ \\ 
		\hypertarget{t}{$t$} & Tiempo & $\text{s}$ \\ 
	\end{tabular}
\end{center}
 % Ruta relativa para la nomenclatura
	
	% Tema n
	\newpage
	% Sección de fórmulas
\titulo{Nombre del tema}
% Ejemplo de fórmulas con enlaces a la nomenclatura

% Añade \hypertarget antes de cada fórmula
\hypertarget{caudal_cont}{}
\formula{Caudal o Ecuación de Continuidad}{$\hyperlink{Q}{Q} = \hyperlink{u}{u} \cdot \hyperlink{A}{A} \rightarrow Q =\dfrac{\hyperlink{V}{V}}{\hyperlink{t}{t}}$}{$\text{Pa}$}



\hypertarget{NOMBRE DEL ENLACE CON EL ARCHIVO NOMENCLATURA.TEX}{}
\formula{Nombre de la fórmula}{$\hyperlink{ENLACE DEL SIMBOLO}{SIMBOLO} = \hyperlink{ENLACE DEL SIMBOLO}{FORMULA}$}{$\text{UNIDADES}$}


%ENLACE CON EL ARCHIVO, NOMBRE DE LA FÓRMULA, LINK DEL SIMBOLO, SIMBOLO, UNIDADES % Ruta relativa para el tema
	
	\newpage
	% Descripción.tex
\sectionnc{Descripción de Fórmulas}

\begin{itemize}
	\item \hyperlink{caudal_cont}{\textbf{Caudal o Ecuación de Continuidad}}: Relaciona el caudal con el área y la velocidad del flujo o el volumen con el tiempo. Úsala para calcular el caudal que pasa por una sección o verificar la conservación del flujo en un sistema.
	
	\item \hyperlink{ENLACE CON ARCHIVO 'Tema n.tex'}{\textbf{NOMBRE DE LA FÓRMULA}}: Lorem ipsum dolor sit amet, consectetur adipiscing elit. Maecenas placerat, eros in viverra euismod, turpis sapien commodo libero, ac varius erat orci sit amet magna. Nam dapibus velit a justo lacinia, non aliquam mauris porta. Ut id sem eu sapien aliquet tincidunt. Aenean congue nec libero ut mattis. Nam varius urna diam. Sed mollis ex a dui scelerisque, non ornare dui dapibus. Proin et libero quis velit accumsan iaculis nec a dolor. Vestibulum fringilla sagittis ante vel lobortis. Cras ullamcorper bibendum pellentesque. Maecenas nisl erat, vulputate sit amet lacus vel, tempor luctus ipsum.
	
\end{itemize} % Ruta relativa para el tema
\end{document}
